\documentclass[dissertation.tex]{subfiles}


\begin{document}


\chapter*{Abstract}\label{chap:abstract}

Data science applications usually revolve around the composition of multiple tasks.
The tasks in such applications are arranged into a \acfp{DAG}, which is widely referred to as a workflow.
Commonly used tools focus on the composition of tasks to form workflows.

The CircuitFlow \ac{DSL} takes an alternative approach by focusing on the data --- not the tasks.
Monoidal resource theories are used to form circuits that describe how data flows through a workflow.
Workflows are described by a symmetric monoidal preorder on data stores, where tasks are preorder relation on data stores.
Workflow are executed using a \acf{KPN}, that exceeds existing libraries for speed, whist being able to propagate failure in the unlikely event a fault occurs.


% This dissertation\todo{or thesis?} explores the composition of tasks using an approach that focuses on \textit{dependencies} between different pieces of data, making use of Monoidal Resource Theories.
% A \ac{DSL} is designed, with constructors that manipulate the paths of data in a workflow, and connects it to tasks that transforms data into new data.
% This DSL forms a \textit{symmetric monoidal preorder} on data stores.



% Abstracts = summary of key lessons and techniques in paper
%  - these are meant to be dense
%  - the audience of these is an expert of the field
%  - the purpose enabling researchers of the future to quickly decide if your work is relevant to what they are researching now
%  - should be about 2 paras - no waffle for nice leading in
%  - just quick tech gorey detail of what this paper contributes
%  - (MAYBE a little bit of motivation)

%  Maybe look at composing tasks with arrows paper?

\end{document}



% ----- Configure Emacs -----
%
% Local Variables: ***
% mode: latex ***
% mmm-classes: literate-haskell-latex ***
% End: ***
