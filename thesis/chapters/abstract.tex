\documentclass[dissertation.tex]{subfiles}


\begin{document}


\chapter*{Abstract}\label{chap:abstract}

% The problem
With the ever growing amount of data collected every day, new approaches are needed to build data science applications.
Its desirable to have assurances that these applications will work correctly, as they can run for days.
However, current dataflow tools are not good enough, too often the interactions between tasks are not made in a type-safe way, leading to undesirable runtime errors.


% Contributions
This thesis presents a new Haskell \acs{e-DSL}, with a strong mathematical basis, refocusing on how data flows through an application, resulting in a more expressive solution that is faster.
Monoidal resource theories are used to define a symmetric monoidal preorder on data that models dependencies in the workflow.
It uses DataKinds and TypeFamilies in phantom type parameters to bake each axiom into the language constructors.
Then using indexed functors and Data types \`{a} la carte to represent the \acs{AST}, and makes use of principled recursion schemes to perform a type-safe translation to a \acf{KPN}.

% Why is it good?


% Commonly used tools focus on the composition of tasks to form workflows.

% The CircuitFlow \ac{DSL} takes an alternative approach by focusing on the data --- not the tasks.
% Monoidal resource theories are used to form circuits that describe how data flows through a workflow.
% Workflows are described by a symmetric monoidal preorder on data stores, where tasks are preorder relation on data stores.
% Workflow are executed using a \acf{KPN}, that exceeds existing libraries for speed, whist being able to propagate failure in the unlikely event a fault occurs.


% \todo[inline]{It's a start... but needs more motivation and the second paragraph is a bit shit}


% This dissertation\todo{or thesis?} explores the composition of tasks using an approach that focuses on \textit{dependencies} between different pieces of data, making use of Monoidal Resource Theories.
% A \ac{DSL} is designed, with constructors that manipulate the paths of data in a workflow, and connects it to tasks that transforms data into new data.
% This DSL forms a \textit{symmetric monoidal preorder} on data stores.


% Abstracts = summary of key lessons and techniques in paper
%  - these are meant to be dense
%  - the audience of these is an expert of the field
%  - the purpose enabling researchers of the future to quickly decide if your work is relevant to what they are researching now
%  - should be about 2 paras - no waffle for nice leading in
%  - just quick tech gorey detail of what this paper contributes
%  - (MAYBE a little bit of motivation)

%  Maybe look at composing tasks with arrows paper?

\end{document}



% ----- Configure Emacs -----
%
% Local Variables: ***
% mode: latex ***
% mmm-classes: literate-haskell-latex ***
% End: ***
