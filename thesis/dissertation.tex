
\documentclass[
author={Riley Evans},
supervisor={Dr. Meng Wang},
degree={MEng},
title={\vbox{Circuit: A Domain Specific Language for Dataflow Programming}},
subtitle={},
type={research},
year={2021}
]{dissertation}
  
%% \usepackage{libertine}
\usepackage{todonotes}
\usepackage{caption}
\usepackage{subcaption}
\usepackage{amsmath}
\usepackage{tikz-cd}
\usepackage{latexsym}

% lhs2tex setup


%% ODER: format ==         = "\mathrel{==}"
%% ODER: format /=         = "\neq "
%
%
\makeatletter
\@ifundefined{lhs2tex.lhs2tex.sty.read}%
  {\@namedef{lhs2tex.lhs2tex.sty.read}{}%
   \newcommand\SkipToFmtEnd{}%
   \newcommand\EndFmtInput{}%
   \long\def\SkipToFmtEnd#1\EndFmtInput{}%
  }\SkipToFmtEnd

\newcommand\ReadOnlyOnce[1]{\@ifundefined{#1}{\@namedef{#1}{}}\SkipToFmtEnd}
\usepackage{amstext}
\usepackage{amssymb}
\usepackage{stmaryrd}
\DeclareFontFamily{OT1}{cmtex}{}
\DeclareFontShape{OT1}{cmtex}{m}{n}
  {<5><6><7><8>cmtex8
   <9>cmtex9
   <10><10.95><12><14.4><17.28><20.74><24.88>cmtex10}{}
\DeclareFontShape{OT1}{cmtex}{m}{it}
  {<-> ssub * cmtt/m/it}{}
\newcommand{\texfamily}{\fontfamily{cmtex}\selectfont}
\DeclareFontShape{OT1}{cmtt}{bx}{n}
  {<5><6><7><8>cmtt8
   <9>cmbtt9
   <10><10.95><12><14.4><17.28><20.74><24.88>cmbtt10}{}
\DeclareFontShape{OT1}{cmtex}{bx}{n}
  {<-> ssub * cmtt/bx/n}{}
\newcommand{\tex}[1]{\text{\texfamily#1}}	% NEU

\newcommand{\Sp}{\hskip.33334em\relax}


\newcommand{\Conid}[1]{\mathit{#1}}
\newcommand{\Varid}[1]{\mathit{#1}}
\newcommand{\anonymous}{\kern0.06em \vbox{\hrule\@width.5em}}
\newcommand{\plus}{\mathbin{+\!\!\!+}}
\newcommand{\bind}{\mathbin{>\!\!\!>\mkern-6.7mu=}}
\newcommand{\rbind}{\mathbin{=\mkern-6.7mu<\!\!\!<}}% suggested by Neil Mitchell
\newcommand{\sequ}{\mathbin{>\!\!\!>}}
\renewcommand{\leq}{\leqslant}
\renewcommand{\geq}{\geqslant}
\usepackage{polytable}

%mathindent has to be defined
\@ifundefined{mathindent}%
  {\newdimen\mathindent\mathindent\leftmargini}%
  {}%

\def\resethooks{%
  \global\let\SaveRestoreHook\empty
  \global\let\ColumnHook\empty}
\newcommand*{\savecolumns}[1][default]%
  {\g@addto@macro\SaveRestoreHook{\savecolumns[#1]}}
\newcommand*{\restorecolumns}[1][default]%
  {\g@addto@macro\SaveRestoreHook{\restorecolumns[#1]}}
\newcommand*{\aligncolumn}[2]%
  {\g@addto@macro\ColumnHook{\column{#1}{#2}}}

\resethooks

\newcommand{\onelinecommentchars}{\quad-{}- }
\newcommand{\commentbeginchars}{\enskip\{-}
\newcommand{\commentendchars}{-\}\enskip}

\newcommand{\visiblecomments}{%
  \let\onelinecomment=\onelinecommentchars
  \let\commentbegin=\commentbeginchars
  \let\commentend=\commentendchars}

\newcommand{\invisiblecomments}{%
  \let\onelinecomment=\empty
  \let\commentbegin=\empty
  \let\commentend=\empty}

\visiblecomments

\newlength{\blanklineskip}
\setlength{\blanklineskip}{0.66084ex}

\newcommand{\hsindent}[1]{\quad}% default is fixed indentation
\let\hspre\empty
\let\hspost\empty
\newcommand{\NB}{\textbf{NB}}
\newcommand{\Todo}[1]{$\langle$\textbf{To do:}~#1$\rangle$}

\EndFmtInput
\makeatother
%
%
%
%
%
%
% This package provides two environments suitable to take the place
% of hscode, called "plainhscode" and "arrayhscode". 
%
% The plain environment surrounds each code block by vertical space,
% and it uses \abovedisplayskip and \belowdisplayskip to get spacing
% similar to formulas. Note that if these dimensions are changed,
% the spacing around displayed math formulas changes as well.
% All code is indented using \leftskip.
%
% Changed 19.08.2004 to reflect changes in colorcode. Should work with
% CodeGroup.sty.
%
\ReadOnlyOnce{polycode.fmt}%
\makeatletter

\newcommand{\hsnewpar}[1]%
  {{\parskip=0pt\parindent=0pt\par\vskip #1\noindent}}

% can be used, for instance, to redefine the code size, by setting the
% command to \small or something alike
\newcommand{\hscodestyle}{}

% The command \sethscode can be used to switch the code formatting
% behaviour by mapping the hscode environment in the subst directive
% to a new LaTeX environment.

\newcommand{\sethscode}[1]%
  {\expandafter\let\expandafter\hscode\csname #1\endcsname
   \expandafter\let\expandafter\endhscode\csname end#1\endcsname}

% "compatibility" mode restores the non-polycode.fmt layout.

\newenvironment{compathscode}%
  {\par\noindent
   \advance\leftskip\mathindent
   \hscodestyle
   \let\\=\@normalcr
   \let\hspre\(\let\hspost\)%
   \pboxed}%
  {\endpboxed\)%
   \par\noindent
   \ignorespacesafterend}

\newcommand{\compaths}{\sethscode{compathscode}}

% "plain" mode is the proposed default.
% It should now work with \centering.
% This required some changes. The old version
% is still available for reference as oldplainhscode.

\newenvironment{plainhscode}%
  {\hsnewpar\abovedisplayskip
   \advance\leftskip\mathindent
   \hscodestyle
   \let\hspre\(\let\hspost\)%
   \pboxed}%
  {\endpboxed%
   \hsnewpar\belowdisplayskip
   \ignorespacesafterend}

\newenvironment{oldplainhscode}%
  {\hsnewpar\abovedisplayskip
   \advance\leftskip\mathindent
   \hscodestyle
   \let\\=\@normalcr
   \(\pboxed}%
  {\endpboxed\)%
   \hsnewpar\belowdisplayskip
   \ignorespacesafterend}

% Here, we make plainhscode the default environment.

\newcommand{\plainhs}{\sethscode{plainhscode}}
\newcommand{\oldplainhs}{\sethscode{oldplainhscode}}
\plainhs

% The arrayhscode is like plain, but makes use of polytable's
% parray environment which disallows page breaks in code blocks.

\newenvironment{arrayhscode}%
  {\hsnewpar\abovedisplayskip
   \advance\leftskip\mathindent
   \hscodestyle
   \let\\=\@normalcr
   \(\parray}%
  {\endparray\)%
   \hsnewpar\belowdisplayskip
   \ignorespacesafterend}

\newcommand{\arrayhs}{\sethscode{arrayhscode}}

% The mathhscode environment also makes use of polytable's parray 
% environment. It is supposed to be used only inside math mode 
% (I used it to typeset the type rules in my thesis).

\newenvironment{mathhscode}%
  {\parray}{\endparray}

\newcommand{\mathhs}{\sethscode{mathhscode}}

% texths is similar to mathhs, but works in text mode.

\newenvironment{texthscode}%
  {\(\parray}{\endparray\)}

\newcommand{\texths}{\sethscode{texthscode}}

% The framed environment places code in a framed box.

\def\codeframewidth{\arrayrulewidth}
\RequirePackage{calc}

\newenvironment{framedhscode}%
  {\parskip=\abovedisplayskip\par\noindent
   \hscodestyle
   \arrayrulewidth=\codeframewidth
   \tabular{@{}|p{\linewidth-2\arraycolsep-2\arrayrulewidth-2pt}|@{}}%
   \hline\framedhslinecorrect\\{-1.5ex}%
   \let\endoflinesave=\\
   \let\\=\@normalcr
   \(\pboxed}%
  {\endpboxed\)%
   \framedhslinecorrect\endoflinesave{.5ex}\hline
   \endtabular
   \parskip=\belowdisplayskip\par\noindent
   \ignorespacesafterend}

\newcommand{\framedhslinecorrect}[2]%
  {#1[#2]}

\newcommand{\framedhs}{\sethscode{framedhscode}}

% The inlinehscode environment is an experimental environment
% that can be used to typeset displayed code inline.

\newenvironment{inlinehscode}%
  {\(\def\column##1##2{}%
   \let\>\undefined\let\<\undefined\let\\\undefined
   \newcommand\>[1][]{}\newcommand\<[1][]{}\newcommand\\[1][]{}%
   \def\fromto##1##2##3{##3}%
   \def\nextline{}}{\) }%

\newcommand{\inlinehs}{\sethscode{inlinehscode}}

% The joincode environment is a separate environment that
% can be used to surround and thereby connect multiple code
% blocks.

\newenvironment{joincode}%
  {\let\orighscode=\hscode
   \let\origendhscode=\endhscode
   \def\endhscode{\def\hscode{\endgroup\def\@currenvir{hscode}\\}\begingroup}
   %\let\SaveRestoreHook=\empty
   %\let\ColumnHook=\empty
   %\let\resethooks=\empty
   \orighscode\def\hscode{\endgroup\def\@currenvir{hscode}}}%
  {\origendhscode
   \global\let\hscode=\orighscode
   \global\let\endhscode=\origendhscode}%

\makeatother
\EndFmtInput
%
%
%
% First, let's redefine the forall, and the dot.
%
%
% This is made in such a way that after a forall, the next
% dot will be printed as a period, otherwise the formatting
% of `comp_` is used. By redefining `comp_`, as suitable
% composition operator can be chosen. Similarly, period_
% is used for the period.
%
\ReadOnlyOnce{forall.fmt}%
\makeatletter

% The HaskellResetHook is a list to which things can
% be added that reset the Haskell state to the beginning.
% This is to recover from states where the hacked intelligence
% is not sufficient.

\let\HaskellResetHook\empty
\newcommand*{\AtHaskellReset}[1]{%
  \g@addto@macro\HaskellResetHook{#1}}
\newcommand*{\HaskellReset}{\HaskellResetHook}

\global\let\hsforallread\empty

\newcommand\hsforall{\global\let\hsdot=\hsperiodonce}
\newcommand*\hsperiodonce[2]{#2\global\let\hsdot=\hscompose}
\newcommand*\hscompose[2]{#1}

\AtHaskellReset{\global\let\hsdot=\hscompose}

% In the beginning, we should reset Haskell once.
\HaskellReset

\makeatother
\EndFmtInput
%



\newmuskip\codemuskip
\codemuskip=4.0mu plus 2.0mu minus 2.0mu\relax
\newcommand\codeskip{\mskip\codemuskip}%
\let\codefont\textsf
\newcommand\sub[1]{\ensuremath{_{\text{#1}}}}

\newcommand\keyw[1]{{\codefont{\textbf{#1}}}}
\newcommand\id[1]{\Varid{#1}}
\newcommand\idsym[1]{\mathbin{\id{#1}}}
\newcommand{\vertrule}[1][1.0ex]{\rule[-0.0ex]{.45pt}{#1}}


\renewcommand\Varid[1]{\codefont{#1}}
\let\Conid\Varid

\begin{document}
  

\maketitle

% =============================================================================

\frontmatter
\makedecl{}
\tableofcontents
\listoftodos

% -----------------------------------------------------------------------------

% \chapter*{Supporting Technologies}

% % -----------------------------------------------------------------------------

% \chapter*{Notation and Acronyms}

% maybe?

% -----------------------------------------------------------------------------

\chapter*{Acknowledgements}

\noindent
It is common practice (although totally optional) to acknowledge any
third-party advice, contribution or influence you have found useful
during your work.  Examples include support from friends or family, 
the input of your Supervisor and/or Advisor, external organisations 
or persons who  have supplied resources of some kind (e.g., funding, 
advice or time), and so on.

% =============================================================================

\mainmatter{}


\chapter{Introduction}\label{chap:intro}

% -----------------------------------------------------------------------------

\chapter{Background}\label{chap:background}

\section{Dataflow Programming}
Dataflow programming is a paradigm that models applications as a directed graph.
The nodes of the graph have inputs and outputs and are pure functions, therefore have no side effects.
It is possible for a node to be a: source; sink; or processing node.
Edges connect these nodes together, and define the flow of information.
\todo[inline]{this feels a little light on detail }


\paragraph{Example - Data Pipelines}
A common use of dataflow programming is in pipelines that process data.
This paradigm is particularly helpful as it helps the developer to focus on each specific transformation on the data as a single component.
Avoiding the need for long and laborious scripts that could be hard to maintain.

\paragraph{Example - Quartz Composer}
Apple developed a tool included in XCode, named Quartz Composer, which is a node-based visual programming language~\cite{quartz}.
It allows for quick development of programs that process and render graphical data.
By using visual programming it allows the user to build programs, without having to write a single line of code.
This means that even non-programmers are able to use the tool.

\paragraph{Example - Spreadsheets}
A widely used example of dataflow programming is in spreadsheets.
A cell in a spreadsheet can be thought of as a single node.
It is possible to specify dependencies to other cells through the use of formulas.
Whenever a cell is updated it sends its new value to those who depend on it, and so on.
Work has also done to visualise spreadsheets using dataflow diagrams, to help debug ones that are complex\cite{hermans2011breviz}.


\subsection{The Benefits}
\paragraph{Visual}
The dataflow paradigm uses graphs, which make programming visual.
It allows the end-user programmer to see how data passes through the program, much easier than in an imperative approach.
In many cases, dataflow programming languages use drag and drop blocks with a graphical user interface to build programs,
for example Tableau Prep~\cite{tableauPrep}.
This makes programming more accessible to users who do not have programming skills.

\paragraph{Implicit Parallelism}
Moore's law states that the number of transistors on a computer chip doubles every two years~\cite{4785860}.
This meant that the chips processing speeds also increased in alignment with Moore's law.
However, in recent years this is becoming harder for chip manufacturers to achieve~\cite{bentley_2020}.
Therefore, chip manufactures have had to turn to other approaches to increase the speed of new chips, such as multiple cores.
It is this approach the dataflow programming can effectively make use of.
Since each node in a dataflow is a pure function, it is possible to parallelise implicitly.
No node can interact with another node, therefore there are no data dependencies outside of those encoded in the dataflow.
Thus eliminating the ability for a deadlock to occur.

\subsection{Dataflow Diagrams}
Dataflow programs are typically viewed as a graph.
An example dataflow graph along with its corresponding imperative approach, is visible in Figure~\ref{fig:dataflow-example}.
In this diagram is possible to see how implicit parallelisation is possible.
Both $A$ and $B$ can be calculated simultaneously, with $C$ able to be evaluated after they are complete.


\begin{figure}[ht]
  \centering
  \begin{subfigure}{0.3\textwidth}
    \centering
    \begin{equation*}
      \begin{aligned}
      A &:= 100 \times X \\
      B &:= X + Y \\
      C &:= A - B \\
      \end{aligned}
    \end{equation*}
    \caption{}
    \label{subfig:dataflow-example-equations}
  \end{subfigure}
  \begin{subfigure}{0.3\textwidth}
    \centering
    \begin{tikzpicture}[node distance={15mm}, main/.style = {draw, thick}]
\node (100) {$100$};
\node (X)   [right of=100] {$X$};
\node (Y)    [right of=X] {$Y$};
\node[main] (plus) [below right of=X, below left of=Y] {$+$};
\node[main] (mul)  [below right of=100, below left of=X] {$\times$};
\node[main] (sub)  [below right of=mul, below left of=plus] {$-$};
\node (C)    [below of=sub] {$C$};

\draw[->, >=stealth] (100) -- (mul);
\draw[->, >=stealth] (X) -- (mul);
\draw[->, >=stealth] (X) -- (plus);
\draw[->, >=stealth] (Y) -- (plus);
\draw[->, >=stealth] (mul) -- (sub);
\draw[->, >=stealth] (plus) -- (sub);
\draw[->, >=stealth] (sub) -- (C);

\end{tikzpicture}

    \caption{}
    \label{subfig:dataflow-example-diagram}
  \end{subfigure}
  \caption{An example dataflow and its imperative approach.}
    \label{fig:dataflow-example}
\end{figure}


\subsection{Kahn Process Networks}
A method introduced by Gilles Kahn, called Kahn Process Networks (KPN) realised this concept through the use of threads
and unbounded FIFO queues~\cite{DBLP:conf/ifip/Kahn74}.
A node in the dataflow becomes a thread in the process network.
These threads are then able to communicate through FIFO queues.
The node can have multiple input queues and is able to read any number of values from them.
It will then compute a result and add it to an output queue.
A requirement of KPNs is that a thread is suspended if it attempts to fetch a value from an empty queue.
It is not possible for a process to test for the presence of data in a queue.

Parks described a variant of KPNs, called Data Processing networks~\cite{381846}.
They recognise that if functions have no side effects then they have no values to be shared between each firing.
Therefore, a pool of threads can be used with a central scheduler instead.




\section{Domain Specific Languages (DSLs)}
A Domain Specific Language (DSL) is a programming language unit that has a specialised domain or use-case.
This differs from a General Purpose Language (GPL), which can be applied across a larger set of domains.
HTML is an example of a DSL, it is good for describing the appearance of websites, however,
it cannot be used for more generic purposes, such as adding two numbers together.

\paragraph{Approaches to Implementation}
DSLs are typically split into two categories: standalone and embedded.
Standalone DSLs require their own compiler and typically their own syntax; HTML would be an example of a standalone DSL.
Embedded DSLs use an existing language as a host, therefore they use the syntax and compiler from the host.
This means that they are easier to maintain and often quicker to develop than standalone DSLs.
An embedded DSL, can be implemented using two differing techniques: shallow and deep embeddings.

\todo[inline]{Add something about why embedded DSLs are used in Haskell}


\subsection{Deep Embeddings}
A deep embedding is when the terms of the DSL will construct an Abstract Syntax Tree (AST) as a host language datatype.
Semantics can then be provided later on with an \ensuremath{\Varid{eval}} function.
Consider the example of a minimal non-deterministic parser combinator library~\cite{wuYoda}.


\begin{hscode}\SaveRestoreHook
\column{B}{@{}>{\hspre}l<{\hspost}@{}}%
\column{3}{@{}>{\hspre}l<{\hspost}@{}}%
\column{12}{@{}>{\hspre}l<{\hspost}@{}}%
\column{30}{@{}>{\hspre}l<{\hspost}@{}}%
\column{E}{@{}>{\hspre}l<{\hspost}@{}}%
\>[B]{}\mathbf{data}\codeskip \Conid{Parser}\codeskip (\Varid{a}\mathbin{::}\Conid{Type})\codeskip \mathbf{where}{}\<[E]%
\\
\>[B]{}\hsindent{3}{}\<[3]%
\>[3]{}\Conid{Satisfy}{}\<[12]%
\>[12]{}\mathbin{::}(\Conid{Char}\to \Conid{Bool})\to \Conid{Parser}\codeskip \Conid{Char}{}\<[E]%
\\
\>[B]{}\hsindent{3}{}\<[3]%
\>[3]{}\Conid{Or}{}\<[12]%
\>[12]{}\mathbin{::}\Conid{Parser}\codeskip \Varid{a}{}\<[30]%
\>[30]{}\to \Conid{Parser}\codeskip \Varid{a}\to \Conid{Parser}\codeskip \Varid{a}{}\<[E]%
\ColumnHook
\end{hscode}\resethooks

\noindent
This can be used to build a parser that can parse the characters \ensuremath{\text{\ttfamily{\textquotesingle}a\textquotesingle}} or \ensuremath{\text{\ttfamily{\textquotesingle}b\textquotesingle}}.

\begin{hscode}\SaveRestoreHook
\column{B}{@{}>{\hspre}l<{\hspost}@{}}%
\column{E}{@{}>{\hspre}l<{\hspost}@{}}%
\>[B]{}\Varid{aorb}\mathbin{::}\Conid{Parser}\codeskip \Conid{Char}{}\<[E]%
\\
\>[B]{}\Varid{aorb}\mathrel{=}\Conid{Satisfy}\codeskip (\equiv \text{\ttfamily{\textquotesingle}a\textquotesingle})\mathbin{\text{\`{}}\Conid{Or}\text{\`{}}}\Conid{Satisfy}\codeskip (\equiv \text{\ttfamily{\textquotesingle}b\textquotesingle}){}\<[E]%
\ColumnHook
\end{hscode}\resethooks

\noindent
However, this parser does not have any semantics, therefore this needs to be provided by the evaluation function \ensuremath{\Varid{parse}}.

\begin{hscode}\SaveRestoreHook
\column{B}{@{}>{\hspre}l<{\hspost}@{}}%
\column{3}{@{}>{\hspre}l<{\hspost}@{}}%
\column{12}{@{}>{\hspre}l<{\hspost}@{}}%
\column{E}{@{}>{\hspre}l<{\hspost}@{}}%
\>[B]{}\Varid{parse}\mathbin{::}\Conid{Parser}\codeskip \Varid{a}\to \Conid{String}\to [\mskip1.5mu (\Varid{a},\Conid{String})\mskip1.5mu]{}\<[E]%
\\
\>[B]{}\Varid{parse}\codeskip (\Conid{Satisfy}\codeskip \Varid{p})\mathrel{=}\lambda \mathbf{case}{}\<[E]%
\\
\>[B]{}\hsindent{3}{}\<[3]%
\>[3]{}[\mskip1.5mu \mskip1.5mu]{}\<[12]%
\>[12]{}\to [\mskip1.5mu \mskip1.5mu]{}\<[E]%
\\
\>[B]{}\hsindent{3}{}\<[3]%
\>[3]{}(\Varid{t}\mathbin{:}\Varid{ts'}){}\<[12]%
\>[12]{}\to [\mskip1.5mu (\Varid{t},\Varid{ts'})\mid \Varid{p}\codeskip \Varid{t}\mskip1.5mu]{}\<[E]%
\\
\>[B]{}\Varid{parse}\codeskip (\Conid{Or}\codeskip \Varid{px}\codeskip \Varid{py})\mathrel{=}\lambda \Varid{ts}\to \Varid{parse}\codeskip \Varid{px}\codeskip \Varid{ts}\plus \Varid{parse}\codeskip \Varid{py}\codeskip \Varid{ts}{}\<[E]%
\ColumnHook
\end{hscode}\resethooks

\noindent
The program can then be evaluated by the \ensuremath{\Varid{parse}} function.
For example, \ensuremath{\Varid{parse}\codeskip \Varid{aorb}\codeskip \text{\ttfamily \char34 a\char34}} evaluates to \ensuremath{}, and \ensuremath{\Varid{parse}\codeskip \Varid{aorb}\codeskip \text{\ttfamily \char34 c\char34}} evaluates to \ensuremath{}.

A key benefit for deep embeddings is that the structure can be inspected, and then modified to optimise the user code: Parsley makes use of such techniques to create optimised parsers~\cite{parsley}.
However, they also have drawbacks - it can be laborious to add a new constructor to the language.
Since it requires that all functions that use the deep embedding be modified to add a case for the new constructor \cite{SVENNINGSSON2015143}.


\subsection{Shallow Embeddings}
In contrast, a shallow approach is when the terms of the DSL are defined as first class components of the language.
For example, a function in Haskell.
Components can then be composed together and evaluated to provide the semantics of the language.
Again a simple parser example can be considered.


\begin{hscode}\SaveRestoreHook
\column{B}{@{}>{\hspre}l<{\hspost}@{}}%
\column{3}{@{}>{\hspre}l<{\hspost}@{}}%
\column{12}{@{}>{\hspre}l<{\hspost}@{}}%
\column{E}{@{}>{\hspre}l<{\hspost}@{}}%
\>[B]{}\mathbf{newtype}\codeskip \Conid{Parser}_{2}\codeskip \Varid{a}\mathrel{=}\Conid{Parser}_{2}\codeskip \{\mskip1.5mu \Varid{parse}_{2}\mathbin{::}\Conid{String}\to [\mskip1.5mu (\Varid{a},\Conid{String})\mskip1.5mu]\mskip1.5mu\}{}\<[E]%
\\[\blanklineskip]%
\>[B]{}\Varid{or}\mathbin{::}\Conid{Parser}_{2}\codeskip \Varid{a}\to \Conid{Parser}_{2}\codeskip \Varid{a}\to \Conid{Parser}_{2}\codeskip \Varid{a}{}\<[E]%
\\
\>[B]{}\Varid{or}\codeskip (\Conid{Parser}_{2}\codeskip \Varid{px})\codeskip (\Conid{Parser}_{2}\codeskip \Varid{py})\mathrel{=}\Conid{Parser}_{2}\codeskip (\lambda \Varid{ts}\to \Varid{px}\codeskip \Varid{ts}\plus \Varid{py}\codeskip \Varid{ts}){}\<[E]%
\\[\blanklineskip]%
\>[B]{}\Varid{satisfy}\mathbin{::}(\Conid{Char}\to \Conid{Bool})\to \Conid{Parser}_{2}\codeskip \Conid{Char}{}\<[E]%
\\
\>[B]{}\Varid{satisfy}\codeskip \Varid{p}\mathrel{=}\Conid{Parser}_{2}\codeskip (\lambda \mathbf{case}{}\<[E]%
\\
\>[B]{}\hsindent{3}{}\<[3]%
\>[3]{}[\mskip1.5mu \mskip1.5mu]{}\<[12]%
\>[12]{}\to [\mskip1.5mu \mskip1.5mu]{}\<[E]%
\\
\>[B]{}\hsindent{3}{}\<[3]%
\>[3]{}(\Varid{t}\mathbin{:}\Varid{ts'}){}\<[12]%
\>[12]{}\to [\mskip1.5mu (\Varid{t},\Varid{ts'})\mid \Varid{p}\codeskip \Varid{t}\mskip1.5mu]){}\<[E]%
\ColumnHook
\end{hscode}\resethooks

\noindent
The same \ensuremath{\Varid{aorb}} parser can be created directly from these functions, avoiding the need for an intermediate AST.

\begin{hscode}\SaveRestoreHook
\column{B}{@{}>{\hspre}l<{\hspost}@{}}%
\column{E}{@{}>{\hspre}l<{\hspost}@{}}%
\>[B]{}\Varid{aorb}_{2}\mathbin{::}\Conid{Parser}_{2}\codeskip \Conid{Char}{}\<[E]%
\\
\>[B]{}\Varid{aorb}_{2}\mathrel{=}\Varid{satisfy}\codeskip (\equiv \text{\ttfamily{\textquotesingle}a\textquotesingle})\mathbin{\text{\`{}}\Varid{or}\text{\`{}}}\Varid{satisfy}\codeskip (\equiv \text{\ttfamily{\textquotesingle}b\textquotesingle}){}\<[E]%
\ColumnHook
\end{hscode}\resethooks


Using a shallow implementation has the benefit of being able add new `constructors' to a DSL, without having to modify any other functions.
Since each `constructor', produces the desired result directly.
However, this causes one of the main disadvantages of a shallow embedding - you cannot inspect the structure.
This means that optimisations cannot be made to the structure before evaluating it.


\section{Higher Order Functors}


It is possible to capture the shape of an abstract datatype as a \ensuremath{\Conid{Functor}}.
The use of a \ensuremath{\Conid{Functor}} allows for the specification of where a datatype recurses.
There is, however, one problem: a \ensuremath{\Conid{Functor}} expressing the parser language is required to be typed.
Parsers require the type of the tokens being parsed.
For example, a parser reading tokens that make up an expression could have the type \ensuremath{\Conid{Parser}\codeskip \Conid{Expr}}.
A \ensuremath{\Conid{Functor}} does not retain the type of a parser.
Instead a type class called \ensuremath{\Conid{IFunctor}} can be used, which is able to maintain the type indicies~\cite{mcbride2011functional}.
This makes use of \ensuremath{\leadsto}, which represents a natural transformation from \ensuremath{\Varid{f}} to \ensuremath{\Varid{g}}.
\ensuremath{\Conid{IFunctor}} can be thought of as a functor transformer: it is able to change the structure of a functor, whilst preserving the values inside it~\cite{lane1998categories}.


\begin{hscode}\SaveRestoreHook
\column{B}{@{}>{\hspre}l<{\hspost}@{}}%
\column{3}{@{}>{\hspre}l<{\hspost}@{}}%
\column{E}{@{}>{\hspre}l<{\hspost}@{}}%
\>[B]{}\mathbf{type}\codeskip (\leadsto)\codeskip \Varid{f}\codeskip \Varid{g}\mathrel{=}\forall \Varid{a}\hsforall \hsdot{\cdot }{.}\Varid{f}\codeskip \Varid{a}\to \Varid{g}\codeskip \Varid{a}{}\<[E]%
\\
\>[B]{}\mathbf{class}\codeskip \Conid{IFunctor}\codeskip \Varid{iF}\codeskip \mathbf{where}{}\<[E]%
\\
\>[B]{}\hsindent{3}{}\<[3]%
\>[3]{}\Varid{imap}\mathbin{::}(\Varid{f}\leadsto\Varid{g})\to \Varid{iF}\codeskip \Varid{f}\leadsto\Varid{iF}\codeskip \Varid{g}{}\<[E]%
\ColumnHook
\end{hscode}\resethooks

\noindent
The shape of \ensuremath{\Conid{Parser}} can be seen in \ensuremath{\Conid{ParserF}} where the \ensuremath{\Varid{f}} marks the recursive spots.
The type \ensuremath{\Varid{f}} represents the type of the children of that node.
In most cases this will be

\begin{hscode}\SaveRestoreHook
\column{B}{@{}>{\hspre}l<{\hspost}@{}}%
\column{3}{@{}>{\hspre}l<{\hspost}@{}}%
\column{13}{@{}>{\hspre}l<{\hspost}@{}}%
\column{E}{@{}>{\hspre}l<{\hspost}@{}}%
\>[B]{}\mathbf{data}\codeskip \Conid{ParserF}\codeskip (\Varid{f}\mathbin{::}\mathbin{*}\to \mathbin{*})\codeskip (\Varid{a}\mathbin{::}\mathbin{*})\codeskip \mathbf{where}{}\<[E]%
\\
\>[B]{}\hsindent{3}{}\<[3]%
\>[3]{}\Conid{SatisfyF}{}\<[13]%
\>[13]{}\mathbin{::}(\Conid{Char}\to \Conid{Bool})\to \Conid{ParserF}\codeskip \Varid{f}\codeskip \Conid{Char}{}\<[E]%
\\
\>[B]{}\hsindent{3}{}\<[3]%
\>[3]{}\Conid{OrF}{}\<[13]%
\>[13]{}\mathbin{::}\Varid{f}\codeskip \Varid{a}\to \Varid{f}\codeskip \Varid{a}\to \Conid{ParserF}\codeskip \Varid{f}\codeskip \Varid{a}{}\<[E]%
\ColumnHook
\end{hscode}\resethooks

\noindent
An \ensuremath{\Conid{IFunctor}} instance can be defined, which follow the same structure as a standard \ensuremath{\Conid{Functor}} instance.

\begin{hscode}\SaveRestoreHook
\column{B}{@{}>{\hspre}l<{\hspost}@{}}%
\column{3}{@{}>{\hspre}l<{\hspost}@{}}%
\column{9}{@{}>{\hspre}l<{\hspost}@{}}%
\column{12}{@{}>{\hspre}l<{\hspost}@{}}%
\column{26}{@{}>{\hspre}l<{\hspost}@{}}%
\column{E}{@{}>{\hspre}l<{\hspost}@{}}%
\>[B]{}\mathbf{instance}\codeskip \Conid{IFunctor}\codeskip \Conid{ParserF}\codeskip \mathbf{where}{}\<[E]%
\\
\>[B]{}\hsindent{3}{}\<[3]%
\>[3]{}\Varid{imap}\codeskip {}\<[9]%
\>[9]{}\anonymous \codeskip {}\<[12]%
\>[12]{}(\Conid{SatisfyF}\codeskip \Varid{s}){}\<[26]%
\>[26]{}\mathrel{=}\Conid{SatisfyF}\codeskip \Varid{s}{}\<[E]%
\\
\>[B]{}\hsindent{3}{}\<[3]%
\>[3]{}\Varid{imap}\codeskip {}\<[9]%
\>[9]{}\Varid{f}\codeskip {}\<[12]%
\>[12]{}(\Conid{OrF}\codeskip \Varid{px}\codeskip \Varid{py}){}\<[26]%
\>[26]{}\mathrel{=}\Conid{OrF}\codeskip (\Varid{f}\codeskip \Varid{px})\codeskip (\Varid{f}\codeskip \Varid{py}){}\<[E]%
\ColumnHook
\end{hscode}\resethooks

\noindent
\ensuremath{\Conid{Fix}} is used to get the fixed point of a \ensuremath{\Conid{Functor}}, to get the indexed fixed point \ensuremath{\Conid{IFix}} can be used.

\begin{hscode}\SaveRestoreHook
\column{B}{@{}>{\hspre}l<{\hspost}@{}}%
\column{E}{@{}>{\hspre}l<{\hspost}@{}}%
\>[B]{}\mathbf{newtype}\codeskip \Conid{Fix}\codeskip \Varid{f}\mathrel{=}\Conid{In}\codeskip (\Varid{f}\codeskip (\Conid{Fix}\codeskip \Varid{f})){}\<[E]%
\\
\>[B]{}\mathbf{newtype}\codeskip \Conid{IFix}\codeskip \Varid{iF}\codeskip \Varid{a}\mathrel{=}\Conid{IIn}\codeskip (\Varid{iF}\codeskip (\Conid{IFix}\codeskip \Varid{iF})\codeskip \Varid{a}){}\<[E]%
\ColumnHook
\end{hscode}\resethooks

\noindent
The fixed point of \ensuremath{\Conid{ParserF}} is \ensuremath{\Conid{Parser}_{3}}.

\begin{hscode}\SaveRestoreHook
\column{B}{@{}>{\hspre}l<{\hspost}@{}}%
\column{E}{@{}>{\hspre}l<{\hspost}@{}}%
\>[B]{}\mathbf{type}\codeskip \Conid{Parser}_{3}\mathrel{=}\Conid{IFix}\codeskip \Conid{ParserF}{}\<[E]%
\ColumnHook
\end{hscode}\resethooks

In a deep embedding, the AST is traversed and modified to make optimisations, however, it may not be the best representation when evaluating it.
This means that it is usually transformed to a different representation. In the case of a parser, this could be a stack machine.
Now that the recursion in the datatype has been generalised, it is possible to create a mechanism to perform this transformation.
An indexed \textit{catamorphism} is one such way to do this, it is a generalised way of folding an abstract datatype.
The commutative diagram below describes how to define a catamorphism, that folds an \ensuremath{\Conid{IFix}\codeskip \Varid{iF}\codeskip \Varid{a}} to a \ensuremath{\Varid{f}\codeskip \Varid{a}}.

\begin{figure}[h]
\centering
\begin{tikzcd}[column sep=huge]
\ensuremath{\Varid{iF}\codeskip (\Conid{IFix}\codeskip \Varid{iF})\codeskip \Varid{a}}  \arrow[r, "\ensuremath{\Varid{imap}\codeskip (\Varid{icata}\codeskip \Varid{alg})}"] \arrow[d, shift left=0.15cm, "\ensuremath{\Conid{IIn}}"] & \ensuremath{\Varid{iF}\codeskip \Varid{f}\codeskip \Varid{a}} \arrow[d, "\ensuremath{\Varid{alg}}"]\\
\ensuremath{\Conid{IFix}\codeskip \Varid{iF}\codeskip \Varid{a}}       \arrow[r, "\ensuremath{\Varid{icata}\codeskip \Varid{alg}}"]        \arrow[u, shift left=0.15cm, "\ensuremath{\Varid{inop}}"]        & \ensuremath{\Varid{f}\codeskip \Varid{a}}
\end{tikzcd}
\end{figure}

\noindent
\ensuremath{\Varid{icata}} is able to fold an \ensuremath{\Conid{IFix}\codeskip \Varid{iF}\codeskip \Varid{a}} and produce an item of type \ensuremath{\Varid{f}\codeskip \Varid{a}}.
It uses the algebra argument as a specification of how to transform a layer of the datatype.

\begin{hscode}\SaveRestoreHook
\column{B}{@{}>{\hspre}l<{\hspost}@{}}%
\column{E}{@{}>{\hspre}l<{\hspost}@{}}%
\>[B]{}\Varid{icata}\mathbin{::}\Conid{IFunctor}\codeskip \Varid{iF}\Rightarrow (\Varid{iF}\codeskip \Varid{f}\leadsto\Varid{f})\to \Conid{IFix}\codeskip \Varid{iF}\leadsto\Varid{f}{}\<[E]%
\\
\>[B]{}\Varid{icata}\codeskip \Varid{alg}\codeskip (\Conid{IIn}\codeskip \Varid{x})\mathrel{=}\Varid{alg}\codeskip (\Varid{imap}\codeskip (\Varid{icata}\codeskip \Varid{alg})\codeskip \Varid{x}){}\<[E]%
\ColumnHook
\end{hscode}\resethooks

\noindent
The resulting type of \ensuremath{\Varid{icata}} is \ensuremath{\Varid{f}\codeskip \Varid{a}}, this requires the \ensuremath{\Varid{f}} to be a \ensuremath{\Conid{Functor}}.
This could be \ensuremath{\Conid{IFix}\codeskip \Conid{ParserF}}, which would be a transformation to the same structure, possibly applying optimisations to the AST.


\section{Data types \`{a} la carte}
When building a DSL one problem that becomes quickly prevalent, the so called \textit{Expression Problem}~\cite{wadler_1998}.
The expression problem is a trade off between a deep and shallow embedding.
In a deep embedding, it is easy to add multiple interpretations to the DSL - just add a new evaluation function.
However, it is not easy to add a new constructor, since all functions will need to be modified to add a new case for the constructor.
The opposite is true in a shallow embedding.

One possible attempt at fixing the expression problem is data types \`{a} la carte.
It combines constructors using the coproduct of their signatures.
This is defined as,


\begin{hscode}\SaveRestoreHook
\column{B}{@{}>{\hspre}l<{\hspost}@{}}%
\column{E}{@{}>{\hspre}l<{\hspost}@{}}%
\>[B]{}\mathbf{data}\codeskip (\Varid{f}:\!\!+\!\!:\Varid{g})\codeskip \Varid{a}\mathrel{=}\Conid{L}\codeskip (\Varid{f}\codeskip \Varid{a})\mid \Conid{R}\codeskip (\Varid{g}\codeskip \Varid{a}){}\<[E]%
\ColumnHook
\end{hscode}\resethooks

\noindent
For each constructor it is possible to define a new data type.

\begin{hscode}\SaveRestoreHook
\column{B}{@{}>{\hspre}l<{\hspost}@{}}%
\column{E}{@{}>{\hspre}l<{\hspost}@{}}%
\>[B]{}\mathbf{data}\codeskip \Conid{Val}\codeskip \Varid{f}\mathrel{=}\Conid{Val}\codeskip \Conid{Int}{}\<[E]%
\\
\>[B]{}\mathbf{data}\codeskip \Conid{Mul}\codeskip \Varid{f}\mathrel{=}\Conid{Mul}\codeskip \Varid{f}\codeskip \Varid{f}{}\<[E]%
\ColumnHook
\end{hscode}\resethooks

\noindent
By using \ensuremath{\Conid{Fix}} to tie the recursive knot, the \ensuremath{\Conid{Fix}\codeskip (\Conid{Val}:\!\!+\!\!:\Conid{Mul})} data type would be isomorphic to a standard \ensuremath{\Conid{Expr}} data type.

\begin{hscode}\SaveRestoreHook
\column{B}{@{}>{\hspre}l<{\hspost}@{}}%
\column{11}{@{}>{\hspre}l<{\hspost}@{}}%
\column{E}{@{}>{\hspre}l<{\hspost}@{}}%
\>[B]{}\mathbf{data}\codeskip \Conid{Expr}\mathrel{=}\Conid{Add}\codeskip \Conid{Expr}\codeskip \Conid{Expr}{}\<[E]%
\\
\>[B]{}\hsindent{11}{}\<[11]%
\>[11]{}\mid \Conid{Val}\codeskip \Conid{Int}{}\<[E]%
\ColumnHook
\end{hscode}\resethooks

\noindent
One problem that now exist, however, is that it is now rather difficult to create expressions, take a simple example of $12 \times 34$.

\begin{hscode}\SaveRestoreHook
\column{B}{@{}>{\hspre}l<{\hspost}@{}}%
\column{E}{@{}>{\hspre}l<{\hspost}@{}}%
\>[B]{}\Varid{exampleExpr}\mathbin{::}\Conid{Fix}\codeskip (\Conid{Val}:\!\!+\!\!:\Conid{Mul}){}\<[E]%
\\
\>[B]{}\Varid{exampleExpr}\mathrel{=}\Conid{In}\codeskip (\Conid{R}\codeskip (\Conid{Mul}\codeskip (\Conid{In}\codeskip (\Conid{L}\codeskip (\Conid{Val}\codeskip \mathrm{12})))\codeskip (\Conid{In}\codeskip (\Conid{L}\codeskip (\Conid{Val}\codeskip \mathrm{34}))))){}\<[E]%
\ColumnHook
\end{hscode}\resethooks

\noindent
It would be beneficial if there was a way to add these \ensuremath{\Conid{L}}s and \ensuremath{\Conid{R}}s automatically. Fortunately there is a method using injections.
The \ensuremath{:\prec:} type class captures the notion of subtypes between \ensuremath{\Conid{Functor}}s.

\begin{hscode}\SaveRestoreHook
\column{B}{@{}>{\hspre}l<{\hspost}@{}}%
\column{3}{@{}>{\hspre}l<{\hspost}@{}}%
\column{E}{@{}>{\hspre}l<{\hspost}@{}}%
\>[B]{}\mathbf{class}\codeskip (\Conid{Functor}\codeskip \Varid{f},\Conid{Functor}\codeskip \Varid{g})\Rightarrow \Varid{f}:\prec:\Varid{g}\codeskip \mathbf{where}{}\<[E]%
\\
\>[B]{}\hsindent{3}{}\<[3]%
\>[3]{}\Varid{inj}\mathbin{::}\Varid{f}\codeskip \Varid{a}\to \Varid{g}\codeskip \Varid{a}{}\<[E]%
\\[\blanklineskip]%
\>[B]{}\mathbf{instance}\codeskip \Conid{Functor}\codeskip \Varid{f}\Rightarrow \Varid{f}:\prec:\Varid{f}\codeskip \mathbf{where}{}\<[E]%
\\
\>[B]{}\hsindent{3}{}\<[3]%
\>[3]{}\Varid{inj}\mathrel{=}\Varid{id}{}\<[E]%
\\[\blanklineskip]%
\>[B]{}\mathbf{instance}\codeskip (\Conid{Functor}\codeskip \Varid{f},\Conid{Functor}\codeskip \Varid{g})\Rightarrow \Varid{f}:\prec:(\Varid{f}:\!\!+\!\!:\Varid{g})\codeskip \mathbf{where}{}\<[E]%
\\
\>[B]{}\hsindent{3}{}\<[3]%
\>[3]{}\Varid{inj}\mathrel{=}\Conid{L}{}\<[E]%
\\[\blanklineskip]%
\>[B]{}\mathbf{instance}\codeskip (\Conid{Functor}\codeskip \Varid{f},\Conid{Functor}\codeskip \Varid{g},\Conid{Functor}\codeskip \Varid{h},\Varid{f}:\prec:\Varid{g})\Rightarrow \Varid{f}:\prec:(\Varid{h}:\!\!+\!\!:\Varid{g})\codeskip \mathbf{where}{}\<[E]%
\\
\>[B]{}\hsindent{3}{}\<[3]%
\>[3]{}\Varid{inj}\mathrel{=}\Conid{R}\hsdot{\cdot }{.}\Varid{inj}{}\<[E]%
\ColumnHook
\end{hscode}\resethooks

\noindent
Using this type class, smart constructors can be defined.

\begin{hscode}\SaveRestoreHook
\column{B}{@{}>{\hspre}l<{\hspost}@{}}%
\column{E}{@{}>{\hspre}l<{\hspost}@{}}%
\>[B]{}\Varid{inject}\mathbin{::}(\Varid{g}:\prec:\Varid{f})\Rightarrow \Varid{g}\codeskip (\Conid{Fix}\codeskip \Varid{f})\to \Conid{Fix}\codeskip \Varid{f}{}\<[E]%
\\
\>[B]{}\Varid{inject}\mathrel{=}\Conid{In}\hsdot{\cdot }{.}\Varid{inj}{}\<[E]%
\\[\blanklineskip]%
\>[B]{}\Varid{val}\mathbin{::}(\Conid{Val}:\prec:\Varid{f})\Rightarrow \Conid{Int}\to \Conid{Expr}\codeskip \Varid{f}{}\<[E]%
\\
\>[B]{}\Varid{val}\codeskip \Varid{x}\mathrel{=}\Varid{inject}\codeskip (\Conid{Val}\codeskip \Varid{x}){}\<[E]%
\\[\blanklineskip]%
\>[B]{}(\mathbin{*})\mathbin{::}(\Conid{Mul}:\prec:\Varid{f})\Rightarrow \Conid{Fix}\codeskip \Varid{f}\to \Conid{Fix}\codeskip \Varid{f}\to \Conid{Fix}\codeskip \Varid{f}{}\<[E]%
\\
\>[B]{}\Varid{x}\mathbin{*}\Varid{y}\mathrel{=}\Varid{inject}\codeskip (\Conid{Mul}\codeskip \Varid{x}\codeskip \Varid{y}){}\<[E]%
\ColumnHook
\end{hscode}\resethooks

\noindent
Expressions can now be built using the constructors, such as \ensuremath{\Varid{val}\codeskip \mathrm{12}\mathbin{*}\Varid{val}\codeskip \mathrm{34}}.


%% \noindent
%% It is also the case that if both |f| and |g| are |Functor|s then so is |f :+: g|.

%% \begin{code}
%% instance (Functor f, Functor g) => Functor (f :+: g) where
%%   fmap f (L x) = L (fmap f x)
%%   fmap f (R y) = R (fmap f y)
%% \end{code}


\section{Type Families}
\begin{itemize}
  \item What are they?
  \item DataKinds
  \item Examples
\end{itemize}

\section{Dependently Typed Programming}
\begin{itemize}
  \item What is is?
  \item Singletons, why they needed, examples, using with typefamilies.
\end{itemize}




% -----------------------------------------------------------------------------

\chapter{Project Execution}\label{chap:execution}



% -----------------------------------------------------------------------------

\chapter{Critical Evaluation}\label{chap:evaluation}

% -----------------------------------------------------------------------------

\chapter{Conclusion}\label{chap:conclusion}

% =============================================================================

\backmatter{}

\bibliography{dissertation}


% =============================================================================

\end{document}

% ----- Configure Emacs -----
%
% Local Variables: ***
% mode: latex ***
% mmm-classes: literate-haskell-latex ***
% End: ***
